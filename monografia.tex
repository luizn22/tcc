\DeclareOption*{\PassOptionsToClass{\CurrentOption}{abntex2}}
\ProcessOptions

\documentclass[
	% -- opções da classe memoir --
	12pt,				% tamanho da fonte
	openright,			% capítulos começam em pág ímpar (insere página vazia caso preciso)
	oneside,			% para impressão em verso e anverso. Oposto a oneside
	a4paper,			% tamanho do papel. 
	% -- opções da classe abntex2 --
	%chapter=TITLE,		% títulos de capítulos convertidos em letras maiúsculas
	%section=TITLE,		% títulos de seções convertidos em letras maiúsculas
	%subsection=TITLE,	% títulos de subseções convertidos em letras maiúsculas
	%subsubsection=TITLE,% títulos de subsubseções convertidos em letras maiúsculas
	% -- opções do pacote babel --
	english,			% idioma adicional para hifenização
	french,				% idioma adicional para hifenização
	spanish,			% idioma adicional para hifenização
	brazil				% o último idioma é o principal do documento
	]{abntex2}

\usepackage{setspace}
\usepackage{conf-ifsc}	
\usepackage{hyperref}
\usepackage{graphicx}
\usepackage{caption}
\usepackage{subcaption}
\usepackage[utf8]{inputenc}
\usepackage{listings}
\usepackage{xcolor}
\usepackage{trivfloat}
\trivfloat{quadro}
\usepackage{enumitem}
\usepackage{xcolor}
\usepackage{multicol}
\usepackage{multirow}
\usepackage{hyperref}
\usepackage{float}
\floatstyle{plaintop}
\restylefloat{quadro}
\usepackage{chngcntr}
\usepackage{lipsum}
\usepackage{tocloft}
\usepackage{tocvsec2}
\usepackage{boxhandler}
\usepackage[font={bf,small},labelfont=small]{caption}
\usepackage{tikz}

\newcommand{\source}[1]{\caption*{\normalfont Fonte: {#1}}}


\counterwithout{equation}{chapter}
\counterwithout{figure}{chapter}
\counterwithout{table}{chapter}

\newcommand\ChangeRT[1]{\noalign{\hrule height #1}}


\definecolor{codegreen}{rgb}{0,0.6,0}
\definecolor{codegray}{rgb}{0.5,0.5,0.5}
\definecolor{codepurple}{rgb}{0.58,0,0.82}
\definecolor{backcolour}{rgb}{0.95,0.95,0.92}

\lstdefinestyle{mystyle}{
    backgroundcolor=\color{backcolour},   
    commentstyle=\color{codegreen},
    keywordstyle=\color{magenta},
    numberstyle=\tiny\color{codegray},
    stringstyle=\color{codepurple},
    basicstyle=\ttfamily\footnotesize,
    breakatwhitespace=false,         
    breaklines=true,                 
    captionpos=b,                    
    keepspaces=true,                 
    numbers=left,                    
    numbersep=5pt,                  
    showspaces=false,                
    showstringspaces=false,
    showtabs=false,                  
    tabsize=2
}

\lstset{style=mystyle}

%---------------------------------------------------------------------%
%---------------------------------------------------------------------%
% Informações de dados para CAPA e FOLHA DE ROSTO
%---------------------------------------------------------------------%
%---------------------------------------------------------------------%

\titulo{DESENVOLVIMENTO DE BIBLIOTECA EM PYTHON PARA AUTOMATIZAÇÃO DOS PROCESSOS DE IDENTIFICAÇÃO E APROXIMAÇÃO DE GANHOS DE CONTROLADOR DE SISTEMAS DE CONTROLE}

\autor{LUIZ FERNANDO NIQUELATTE}

\local{FLORIANÓPOLIS}

\data{2023.}

\orientador[Orientador:\\]{Prof. Cynthia Beatriz Scheffer Dutra, Prof.Dra.Eng.}


%\coorientador[Coorientador:\\]{Nome do coorientador}

\tipotrabalho{Monografia (Graduação)}

% O preambulo deve conter o tipo do trabalho, o objetivo, o nome da instituição e a área de concentração 
\preambulo{Trabalho de Conclusão de Curso submetido ao Instituto Federal de Educação, Ciência e Tecnologia de Santa Catarina como parte dos requisitos para obtenção do título de Engenheira Mecatrônica.}

% \textoaprovacao{Este Trabalho foi julgado adequado para obtenção do Título de Engenheira Eletrônica em abril de 2021 e aprovado na sua forma final pela banca examinadora do Curso de Engenharia Eletrônica do instituto Federal de Educação Ciência, e Tecnologia de Santa Catarina.}


%---------------------------------------------------------------------%
% Início do documento
%---------------------------------------------------------------------%

\begin{document}



\selectlanguage{brazil}
\frenchspacing 


% ----------------------------------------------------------
% ELEMENTOS PRÉ-TEXTUAIS
% ----------------------------------------------------------
% \pretextual

\imprimircapa
\imprimirfolhaderosto* %(o * indica que haverá a ficha bibliográfica)

%---------------------------------------------------------------------%
% ATENÇÃO - Pergunte para a Biblioteca do IFSC
% Inserir a ficha bibliografica - 
%
% Para gerar a ficha catalográfica acesse:
% http://ficha.florianopolis.ifsc.edu.br/
% Precisa ser feito pelo navegador Mozilla Firefox
%---------------------------------------------------------------------%

\imprimirficha{pdf/fichacatalografica.pdf}
%\cleardoublepage

%---------------------------------------------------------------------%
% Inserir folha de aprovação
%---------------------------------------------------------------------%


%\imprimiraprovacao
\includepdf[pages=-]{pdf/tcc-aprovado.pdf}

%\cleardoublepage

%---------------------------------------------------------------------%
% Dedicatória
%---------------------------------------------------------------------%
%\begin{dedicatoria}
%    \vspace*{\fill}
%	\begin{flushright}
%    		(Dedicatória é um elemento opcional.\\
%            Texto alinhado no canto inferior direito.\\
%            Não deve ultrapassar uma página.)
%	\end{flushright}
%\end{dedicatoria}




%---------------------------------------------------------------------%
% Agradecimentos
%---------------------------------------------------------------------%
\begin{agradecimentos}
    Elemento opcional que não pode ultrapassar o limite de uma página.
\end{agradecimentos}
% ---

%---------------------------------------------------------------------%
% Epígrafe
%---------------------------------------------------------------------%
%\begin{epigrafe}
%    \vspace*{\fill}
%	\begin{flushright}
%    		(Epígrafe é um elemento opcional.\\
%    		Texto alinhado no canto inferior direito.\\
%            Não deve ultrapassar uma página.)
%	\end{flushright}
%\end{epigrafe}

%---------------------------------------------------------------------%
% RESUMOS
%---------------------------------------------------------------------%
% resumo em português
\setlength{\absparsep}{18pt} % ajusta o espaçamento dos parágrafos do resumo
\renewcommand{\baselinestretch}{1} 
\begin{resumo}
    No campo de controle de sistemas, a configuração de controladores PID em ambientes dinâmicos representa um desafio
    técnico significativo.
    Visando contribuir para a simplificação deste processo, este trabalho aborda o desenvolvimento de uma biblioteca em
    Python, para automatizar os processos de identificação e aproximação de ganhos de controladores PID em sistemas de
    controle.
    A ferramenta proposta visa oferecer uma ferramenta eficaz e de fácil uso, projetada para ser uma contribuição
    valiosa tanto para profissionais quanto para estudantes e pesquisadores na área de controle e automação.
    A biblioteca foi arquitetada enfatizando a compreensibilidade e a escalabilidade, visando eficácia imediata e a
    capacidade de adaptação a necessidades futuras.
    A integração com a biblioteca de controle em Python existente amplia as funcionalidades da nova ferramenta,
    aumentando sua versatilidade.
    Além disso, o design da biblioteca é orientado para usuários com menos experiência em programação Python,
    facilitando a aplicação prática dos conceitos teóricos no campo do controle e automação.
 
   \noindent 
    \textbf{Palavras-chave}: Python. Sistemas de Controle. Automatização. Identificação. Controlador PID.
\end{resumo}

% resumo em inglês
\renewcommand{\baselinestretch}{1} 
\begin{resumo}[Abstract]
 \begin{otherlanguage*}{english}

   In the field of control systems, configuring PID controllers in dynamic environments represents a significant
   technical challenge.
   Aiming to contribute to the simplification of this process, this work addresses the development of a Python library
   for automating the processes of identification and approximation of gains of PID controllers in control systems.
   The proposed tool aims to provide an effective and easy-to-use resource, designed to be a valuable contribution for
   both professionals and students and researchers in the field of control and automation.
   The library has been architected with an emphasis on comprehensibility and scalability, aiming for immediate efficacy
   and the ability to adapt to future needs.
   Integration with the existing Python control library expands the functionalities of the new tool, increasing its
   versatility.
   Furthermore, the design of the library is oriented towards users with less experience in Python programming,
   facilitating the practical application of theoretical concepts in the field of control and automation.

   %\vspace{-0.8cm}
 
   \noindent 
   \textbf{Keywords}: Python. Control Systems. Automation. Identification. PID Controller.
 \end{otherlanguage*}
\end{resumo}


%---------------------------------------------------------------------%
% inserir lista de ilustrações
%---------------------------------------------------------------------%
\renewcommand{\listfigurename}{Lista de Figuras}
\pdfbookmark[0]{\listfigurename}{lof}
\listoffigures*
\cleardoublepage

%---------------------------------------------------------------------%
% inserir lista de quadros
%---------------------------------------------------------------------%

\renewcommand{\listquadroname}{Lista de quadros}
\newfloat{quadro}{\quadroname}{loq}[chapter]
\setfloatlocations{quadro}{hbtp}
\newlistof{listofquadros}{loq}{\listquadroname}
\newlistentry{quadro}{loq}{0}
\renewcommand{\cftquadroname}{\quadroname\space}
\renewcommand*{\cftquadroaftersnum}{\hfill\textendash\hfill}
\counterwithout{quadro}{chapter}
\listofquadros* 
\cleardoublepage



%---------------------------------------------------------------------%
% inserir lista de tabelas
%---------------------------------------------------------------------%
\pdfbookmark[0]{\listtablename}{lot}
\listoftables*
\cleardoublepage

%---------------------------------------------------------------------%
% inserir lista de listings
%---------------------------------------------------------------------%
%\pdfbookmark[0]{\lstlistlistingname}{lol}
%\listoflistings
%\cleardoublepage

%---------------------------------------------------------------------%
% inserir lista de abreviaturas e simbolos
%---------------------------------------------------------------------%
%\listofabrev{tex/00-Abreviaturas}
\imprimirlistadeabreviaturas

% \imprimirlistadesimbolos
\cleardoublepage

%---------------------------------------------------------------------%
% inserir o sumario

%---------------------------------------------------------------------%


\setlength{\cftbeforechapterskip}{4pt plus 0pt}
\pdfbookmark[0]{\contentsname}{toc}
\tableofcontents *
\cleardoublepage

% ----------------------------------------------------------
% ELEMENTOS TEXTUAIS
% ----------------------------------------------------------
\textual

% ----------------------------------------------------------
% Inclusão dos capítulos que estão em outros arquivos .tex
% ----------------------------------------------------------

\chapter{Introdução}

\abreviatura{PID}{Proporcional-Integral-Derivativo}

O desenvolvimento tecnológico na área de controle de sistemas tem experimentado um crescimento significativo nas
últimas décadas.
Os controladores Proporcional-Integral-Derivativo (PID) são utilizados na maioria das aplicações de controle de
processos automáticos na indústria atualmente, regulando fluxo, temperatura, pressão, nível e muitas outras variáveis
de processos industriais.
Esses controladores, que datam de 1939, são a base dos sistemas modernos de controle de processos, automatizando tarefas
de regulação que, de outra forma, teriam que ser feitas manualmente \cite{introart1}.

Apesar de sua ampla utilização, a configuração eficaz de um controlador PID permanece um desafio, especialmente em
sistemas com dinâmicas complexas e variáveis.
A sintonização desses controladores exige um entendimento profundo do sistema a ser controlado e habilidades
especializadas para ajustar os parâmetros adequadamente \cite{introart2}.

A implementação de controladores PID, embora bem estabelecida, envolve processos que podem ser complexos e demorados,
especialmente na identificação de modelos e na aproximação de ganhos.
A integração de tecnologias computacionais modernas, como a programação em Python, tem aberto caminhos para a
realização desses processos, possibilitando realizar eles de forma assistida por software.
Ao empregar algoritmos e softwares especializados, é possível simplificar e agilizar as etapas tradicionais de
identificação de modelo e configuração dos controladores PID, tornando-os mais acessíveis e eficientes, especialmente
em ambientes industriais com sistemas dinâmicos.

Neste contexto, a automação e a otimização dos processos de identificação de sistemas, sintonização e modelagem de
controladores PID através de ferramentas computacionais tornam-se uma ferramenta útil, visto que podem proporcionar
uma dinâmica mais ágil a esses processos, possibilitando a análise de mais formas de identificação e parâmetros de
ganho PID, além de reduzir erros de cálculo ou aplicação de método.

\section{Justificativa}

A automatização no processo de identificação e sintonização de controladores PID oferecida por uma biblioteca em Python
pode representar um avanço interessante na eficiência e precisão desses processos.
A complexidade e o tempo necessários para a sintonização eficaz são podem ser reduzidos, e erros
associados às abordagens manuais ou menos automatizadas também podem ser minimizados.

Outra vantagem significativa é a construção da biblioteca em Python utilizando a biblioteca de
controle para Python já existente, a \textit{Python Control Systems Library} em vez de ferramentas como MATLAB.
Python, sendo uma linguagem de programação gratuita e de código aberto, oferece várias vantagens sobre o MATLAB, que é
proprietário e licenciado individualmente \cite{introart3}.
Além de ser uma opção mais econômica, o Python permite uma maior transparência e colaboração dentro da comunidade de
desenvolvedores, graças à sua natureza de código aberto.

A relevância deste projeto está em fornecer uma ferramenta de software de código aberto que possa ser utilizada na área
de controle.
A biblioteca desenvolvida pode ser uma contribuição relevante para a comunidade profissional, possivelmente promovendo
o avanço da área de controle, até mesmo de sistemas mais complexos, e facilitando a aplicação prática desses
conhecimentos em diversos setores da indústria.

Por fim, existe também o potencial significativo para uso didático com a documentação detalhada dos métodos
implementados e o acesso fácil proporcionado pela linguagem Python.
Isso pode tornar a biblioteca uma ferramenta valiosa, tanto para educadores quanto para estudantes no campo da
engenharia de controle, facilitando o aprendizado e a experimentação com conceitos de controle PID em um ambiente mais
acessível e flexível.

\section{Objetivo Geral}

O objetivo central deste trabalho é desenvolver uma biblioteca em Python para análise
e controle de sistemas de primeira ordem, visando agilizar e facilitar o processo de
análise e desenvolvimento de sistemas de controle.

\section{Objetivos Específicos}

Para atingir o objetivo principal, os seguintes itens se fazem importantes e necessários:
\begin{alineas}
    \item Criar uma biblioteca em python disponível para instalação em ambientes Windows e Linux;
    \item Utilizar uma arquitetura de código compreensível e escalável, a fim de facilitar implementações futuras
    que aumentem as funcionalidades da biblioteca;
    \item Usar a biblioteca de controle para Python já existente nas implementações, possibilitando o uso dos modelos
    criados para fins não previstos pela biblioteca;
    \item Projetar a biblioteca com facilidade de uso em mente, para que mesmo estudantes ou pessoas pouco instruídas
    em ferramentas da linguagem Python possam fazer uso da biblioteca;
    \item Desenvolver métodos de identificação de modelo da planta, com o mínimo de ações e informações necessárias por
    parte do usuário;
    \item Implementar funções para obtenção de ganhos de controle PID aproximados, de forma automática ou simplificada,
    com base no modelo da planta e em metas de desempenho do sistema;
    \item Fornecer ferramentas para visualização dos dados, desempenho e análise gráfica dos resultados;
    \item Documentar o uso e funcionamento da biblioteca de forma compreensível e acessível.
\end{alineas}

\chapter{Fundamentação Teórica}


\section{Sistemas de Controle}

Os sistemas de controle constituem um conjunto integrado de componentes que visam a regulação e a supervisão do
comportamento de sistemas dinâmicos.
Essenciais em uma muitas de aplicações práticas, são empregados para assegurar a estabilidade operacional e a precisão
de resposta a perturbações.
A operação desses sistemas pode ser realizada sob duas configurações principais: sistemas de malha aberta,
onde não há realimentação do estado do sistema, e sistemas de malha fechada, que se caracterizam pela incorporação de
realimentação na estratégia de controle.
A eficiência de um sistema de controle é determinada pela sua habilidade em alcançar e sustentar um estado operacional
desejado, reduzindo desvios e oscilações indesejadas.
A precisão com que um sistema de controle atinge e mantém a saída desejada, apesar das flutuações nos parâmetros ou nas
condições ambientais, é um indicador crítico de seu desempenho. \cite{ogata2010engenharia}.

Um sistema de controle em malha aberta opera sem a leitura de qualquer variável.
Nesse arranjo, o controlador atua baseado apenas no sinal de entrada, sem ajustar sua ação em resposta a distúrbios ou
variações na saída.
Um exemplo clássico é o controle de velocidade de um motor, onde a quantidade de combustível é ajustada apenas com base
na velocidade desejada, sem considerar a velocidade real do motor. \cite[Cap 2.3]{ogata2010engenharia}.

Em contraste, um sistema de controle em malha fechada inclui um ciclo de realimentação, onde a saída é continuamente
monitorada e comparada com o sinal de referência.
A diferença entre esses dois, conhecida como erro, é utilizada pelo controlador para ajustar o sinal de controle e,
assim, minimizar o erro. \cite[Cap 2.3]{ogata2010engenharia}.

\begin{figure}[H]
    \centering
    \caption{Diagrama de blocos de um sistema de malha fechada}
    \includegraphics[scale=1]{figuras/closed_loop}
    \label{fig:closed_loop}
    \\
    \vspace{0cm}\hspace{0cm}\small{Fonte: \cite[Fig 2.3]{ogata2010engenharia}}
\end{figure}

Onde:
\begin{itemize}
    \item $R(s)$: Representa o sinal de referência.
    \item $G(s)$: Representa o sistema sistema.
    \item $C(s)$: Representa a saida.
    \item $E(s)$: É o erro ou a diferença entre $R(s)$ e $C(s)$.
\end{itemize}

\subsection{Modelo}

De acordo com \cite{CoelhoIdentificacao}, em controle de processos, um modelo não busca ser uma réplica exata do
sistema real, mas sim uma representação adequada para uma aplicação específica.
A modelagem é um procedimento que visa obter um conjunto de equações matemáticas que descrevem a dinâmica do sistema,
permitindo responder a questões sobre o sistema sem a realização de experimentações físicas.
A simplicidade é muitas vezes uma virtude na modelagem de processos, pois modelos excessivamente complexos podem
não ser necessários para capturar a dinâmica essencial do sistema para fins de controle.

A função de transferência é uma forma comum de representar o modelo de um sistema em controle de processos.
Ela é definida como a razão entre a transformada de Laplace da saída e da entrada do sistema,
sendo uma razão de dois polinômios em $s$, onde $s$ é a variável complexa da transformada de Laplace.
Esta representação é particularmente útil para a análise e o projeto de sistemas de controle,
pois permite uma avaliação clara da resposta do sistema a diferentes tipos de sinais de entrada,
como impulso, degrau, rampa e senoidal.

Um modelo clássico e comumente utilizado para representação de sistemas de controle de primeira ordem com atraso é
representado pela seguinte função de transferência:
\begin{equation}
    \label{eq:firstordertf}
    G(s) = \frac{K}{\tau s + 1}e^{-\theta s}
\end{equation}
onde cada termo tem um significado específico:
\begin{itemize}
    \item $G(s)$: Função de transferência do sistema no domínio da frequência.
    \item $K$: Ganho do sistema, que determina a amplitude da saída em relação à entrada.
    \item $\tau$: Constante de tempo do sistema, que indica a rapidez com que o sistema responde a uma entrada.
    \item $\theta$: Tempo de atraso, que representa o tempo que leva para a resposta do sistema começar após uma entrada.
    \item $s$: Variável complexa da Transformada de Laplace, usada para transformar funções do tempo para o domínio da frequência.
\end{itemize}

Este modelo é particularmente útil para descrever sistemas onde há um atraso perceptível entre a ação de controle e a
resposta observada.
A exponencial negativa \( e^{-\theta s} \) incorpora o atraso no modelo, deslocando a resposta do sistema no tempo.
A constante de tempo \( \tau \) e o ganho \( K \) são parâmetros fundamentais que influenciam a dinâmica do sistema.
Através de simulações baseadas neste modelo, é possível prever o comportamento do sistema sob diferentes
condições operacionais e ajustar o projeto de um controlador antes da implementação real.

No contexto da automação industrial, os modelos matemáticos são empregados para previsão, análise e projeto de sistemas
de controle, essenciais para a sintonia de controladores e a otimização de processos.

\subsection{Controlador}

Dentro do universo dos sistemas de controle, um controlador é um componente crucial que modula a entrada de um sistema
para alcançar a saída desejada.
Ele atua ajustando o sinal de controle em resposta às variações da saída, visando minimizar a diferença entre a saída
observada e a saída desejada, conhecida como sinal de referência.
Os controladores podem ser classificados de acordo com suas ações de controle, alguns exemplos são controladores,
como on-off, proporcionais, integrais, proporcional-integrais (PI), proporcional-derivativos (PD) e
proporcional-integral-derivativos (PID), cada um com características distintas que os tornam adequados para diferentes
aplicações industriais. \cite[Cap 2.3]{ogata2010engenharia}.

%chck
O controlador PID é um dos tipos mais prevalentes de controladores em sistemas de controle, caracterizado pela sua
função de transferência,
\begin{equation}
    \label{eq:ctrlr}
    G_c(s) = K_p + \frac{K_i}{s} + K_d s
\end{equation}
onde \( K_p \), \( K_i \), e \( K_d \) representam os ganhos proporcional, integral e derivativo, respectivamente.
O termo proporcional \( K_p \) determina a reação do controlador à magnitude atual do erro,
o termo integral \( K_i \) acumula o erro ao longo do tempo, visando eliminar o erro estático,
e o termo derivativo \( K_d \) responde à taxa de variação do erro, antecipando o comportamento futuro.
A escolha adequada desses parâmetros é crucial: um \( K_p \) elevado pode acelerar a resposta do sistema, mas
potencialmente à custa da estabilidade;
um \( K_i \) excessivo pode introduzir oscilações devido ao atraso na resposta;
e um \( K_d \) significativo pode melhorar a estabilidade e a resposta rápida, mas é sensível ao ruído do sinal de
medição.
O ajuste dos três efeitos do controlador PID é essencial para otimizar o desempenho do sistema, tanto em resposta
transitória quanto em regime estacionário. \cite[Cap 2.3 e 8]{ogata2010engenharia}.

Vale ressaltar que controladores como P, PI e PD, podem ser vistos como um controlador PID onde o ganho para os
parâmetros não citados é ajustado para zero.

\subsection{Métodos de Identificação}

Segundo \cite[Cap 2]{CoelhoIdentificacao}, métodos de identificação de sistemas referem-se ao conjunto de técnicas
utilizadas para construir modelos matemáticos que capturam a dinâmica de sistemas reais a partir de dados de entrada e
saída.
Estes métodos buscam determinar os parâmetros do sistema que melhor se ajustam às medidas observadas,
permitindo que o modelo matemático reproduza o comportamento do sistema.
A identificação pode ser conduzida de duas formas principais: \textit{off-line} e \textit{on-line}.

Na identificação \textit{off-line}, coleta-se um conjunto de dados de entrada e saída, comumente referidos como dados discretos,
processados posteriormente para estimar os parâmetros do modelo.
Este processo não possui restrições de tempo computacional e é tipicamente realizado utilizando algoritmos
não-recursivos.
Já a identificação \textit{on-line} ajusta os parâmetros do modelo em tempo real, sem a necessidade de armazenar previamente as
medidas.
Utiliza-se um algoritmo recursivo que atualiza os parâmetros após cada amostra coletada,
o que é particularmente útil para sistemas que mudam com o tempo ou quando é necessário um ajuste contínuo do modelo.

Devido ao escopo deste trabalho, métodos de identificação \textit{on-line} não serão abordados, apenas métodos \textit{off-line} baseados
nos dados discretos de resposta do sistema.

\subsubsection{Ziegler Nichols}\label{subsubsec:znfun}

O método Ziegler-Nichols é um clássico da literatura para identificação de modelos dinâmicos de
processos industriais.
Desenvolvido por Ziegler e Nichols em 1942 \cite[Cap 4]{CoelhoIdentificacao}.

O processo de identificação de um modelo por esse método envolve a análise da resposta de um sistema a um sinal
degrau, para obter os parâmetros da equação\eqref{eq:firstordertf}, $K$, $\tau$ e $\theta$ e criar
a função de transferência representativa do modelo do sistema analisado.

Os parâmetros são calculados conforme indicado pela figura a seguir:
\begin{figure}[H]
    \centering
    \caption{Métodos de ZN e HAG para a modelagem de processos de primeira ordem}
    \includegraphics[scale=0.3]{figuras/zn_hg_ident_meth}
    \label{fig:zn_hg_ident_meth}
    \\
    \vspace{0cm}\hspace{0cm}\small{Fonte: \cite[Fig 4.3]{CoelhoIdentificacao}}
\end{figure}

A reta traçada corresponde à tangente no ponto de máxima inclinação da curva de reação.

A constante de tempo $\tau$ é determinada pelo intervalo de tempo entre $t_1$, e o instante
$t_3$, onde a reta tangente toca o eixo $t$, e onde cruza com a reta $y(t) = y_f$,
respectivamente.
O valor de  $\theta$ é considerado como $t_1$, o intervalo entre a aplicação do sinal degrau e o
momento em que a reta tangente toca o eixo $t$.
Por fim o valor de $K$ pode ser obtido a través da equação,
\begin{equation}
    \label{eq:dydu}
    K = \frac{\Delta y}{\Delta u}
\end{equation}
Ou seja $y_f$ dividido pelo valor do sinal degrau. \cite[Cap 4]{CoelhoIdentificacao}.

\subsubsection{Hagglund}

O método Hagglund é mais um clássico da literatura para identificação de modelos dinâmicos de
processos industriais.
Desenvolvido por Hagglund em 1991 \cite[Cap 4]{CoelhoIdentificacao}.

A identificação de modelo pelo método de Hägglund é muito similar a de Ziegler e Nichols (\ref{subsubsec:znfun}), com a
única diferença sendo que $\tau$ é determinado pelo intervalo de tempo entre $t_1$, e o instante $t_2$, sendo $t_2$
o momento em que a curva de resposta alcança o valor $y(t) = y(0) + 0.632y_f$, ou seja $63.2\%$ do valor de regime.
Isso pode ser visualizado na figura \ref{fig:zn_hg_ident_meth}.

\subsubsection{Smith}\label{subsubsec:smfun}

Desenvolvido por Smith em 1985, o método Smith, assim como os métodos Hagglund e Ziegler-Nichols, busca encontrar
valores para os parâmetros $K$, $\tau$ e $\theta$ e criar a função de transferência representativa do modelo do sistema
analisado \cite[Cap 4]{CoelhoIdentificacao}.

\begin{figure}[H]
    \centering
    \caption{Método de Smith para a modelagem de processos de primeira ordem}
    \includegraphics[scale=0.3]{figuras/sm_ident_meth}
    \label{fig:sm_ident_meth}
    \\
    \vspace{0cm}\hspace{0cm}\small{Fonte: \cite[Fig 4.3]{CoelhoIdentificacao}}
\end{figure}

Na figura \ref{fig:sm_ident_meth} podem ser observados os momentos $t1$ e $t2$, eles correspondem a passagem da resposta
pelos pontos $y(t) = y(0) + 0.283y(\infty)$ e $y(t) = y(0) + 0.632y(\infty)$, respectivamente.

As constantes podem então ser utilizadas para o cálculo da constante de tempo $\tau$ e do valor de $\theta$,
conforme as seguintes equações:
\begin{equation}
    \label{eq:smtau}
    \tau = 1.5*(t_2 - t_1)
\end{equation}
\begin{equation}
    \label{eq:smtheta}
    \theta = t_2 - \tau
\end{equation}

Por fim o valor de $K$ pode ser obtido através da equação \eqref{eq:dydu}.

\subsubsection{Sundaresan Krishnaswamy}

O método de Sundaresan e Krishnaswamy, desenvolvido em 1977 é bastante similar ao método Smith \ref{subsubsec:smfun}.
Apresentando algumas alterações nos valores para o cácluculo das constantes de tempo $t1$ e $t2$, calculadas como
$y(t) = y(0) + 0.353y(\infty)$ e $y(t) = y(0) + 0.853y(\infty)$, respectivamente, como pode ser visto na figura
\ref{fig:sd_kr_ident_meth} \cite[Cap 4]{CoelhoIdentificacao}:


\begin{figure}[H]
    \centering
    \caption{Método de Sundaresan e Krishnaswamy para a modelagem de processos de primeira ordem }
    \includegraphics[scale=0.3]{figuras/sd_kr_ident_meth}
    \label{fig:sd_kr_ident_meth}
    \\
    \vspace{0cm}\hspace{0cm}\small{Fonte: \cite[Fig 4.3]{CoelhoIdentificacao}}
\end{figure}

Além de ter diferenças nas equações de $\tau$ e $\theta$, como pode ser visto nas equações \eqref{eq:sktau} e
\eqref{eq:sktheta}:
\begin{equation}
    \label{eq:sktau}
    \tau = 0.67*(t_2 - t_1)
\end{equation}
\begin{equation}
    \label{eq:sktheta}
    \theta = 1.3t_1 - 0.29t_2
\end{equation}

\subsubsection{Nishikawa}

Um último um método clássico da literatura para Identificação de modelos é o desenvolvido por Nishikawa em 1984.
A pesar de buscar os mesmos parâmetros que os métodos anteriores, estes parâmetros são obtidos através de áreas
delimitadas pela curva de resposta a sinal degrau, pelo valor de regime e pela constante de tempo $t_0$, como pode ser
observado na figura \ref{fig:ni_ident_meth}.


\begin{figure}[H]
    \centering
    \caption{Método de Nishikawa para a modelagem de processos de primeira ordem}
    \includegraphics[scale=0.3]{figuras/ni_ident_meth}
    \label{fig:ni_ident_meth}
    \\
    \vspace{0cm}\hspace{0cm}\small{Fonte: \cite[Fig 4.3]{CoelhoIdentificacao}}
\end{figure}

As áreas e a constante de tempo podem ser calculadas conforme as seguintes equações:

\begin{equation}
    \label{eq:nia0}
    A_0 = \int_{0}^{\infty} { \Delta y(\infty) - \Delta y(t) } dt
\end{equation}
\begin{equation}
    \label{eq:nia1nt0}
    A_1 = \int_{0}^{t_0} \Delta y(t) dt \;\; ; \;\; t_0 = \frac{A_0}{\Delta y(\infty)}
\end{equation}

Com isto, podem ser obtodas as constantes $\tau$ e $\theta$:
\begin{equation}
    \label{eq:nitau}
    \tau = \frac{A_1}{0.368\Delta y(\infty)}
\end{equation}
\begin{equation}
    \label{eq:nitheta}
    \theta = t_0 - \tau
\end{equation}

Da mesma forma que os outros métodos, $K$ pode ser obtido através da equação \eqref{eq:dydu}.

\subsection{Métodos de Aproximação de Controlador PID}

A sintonia de controladores PID é um componente crucial no desenvolvimento de sistemas de controle automatizados,
fundamental para assegurar a eficiência e a estabilidade operacional.
A importância da sintonia reside na sua capacidade de ajustar o comportamento do sistema de controle para atender às
especificações de desempenho, como precisão, rapidez de resposta e estabilidade a longo prazo.
A sintonia adequada consegue compensar as incertezas inerentes ao modelo do sistema e as variações ambientais,
garantindo que o sistema mantenha seu desempenho ótimo sob uma gama de condições operacionais.
\cite{apostpidsint}.

Realizar a sintonia de um controlador PID envolve a aproximação dos ganhos do controlador capazes de causar o efeito no
sistema.
Para isso é comumente feita a análise o modelo do sistema a ser controlado.
Através da análise do modelo, é possível prever como o sistema responde a diferentes configurações de controle e
identificar os parâmetros de ganho que resultarão na resposta desejada.
Este processo de aproximação dos ganhos, permite a sintonia do sistema, assegurando que o controlador PID opere de forma
eficiente, mantendo a saída do sistema nos parâmetros desejados, minimizando o erro e otimizando a resposta a
perturbações. \cite{apostpidsint}.

\subsubsection{Ziegler Nichols}\label{subsubsec:znctr}

Proposto por Ziegler e Nichols, como uma forma de obter os ganhos de controlador para os modelos itentificados
pelo seu método de indentificação (\ref{subsubsec:znfun}), o objetivo deste método é obter parâmetros de ganho PID que
façam a sintonia do controlador \cite{apostpidsint}.

Em espessífico, este método se baseia na curva de reação do sistema a resposta de sinal degrau, que é
exatamente o que o resultado da identificação por Ziegler Nichols obtém, utilizando os parâmetros $K$, $\tau$ e $\theta$
para análise.

Desta forma, são aplicadas as fórmulas do método de acordo com a tabela \ref{tab:zncntb}.

\begin{table}[h]
    \begin{center}
        \begin{tabular}{ | l | c | c | c | }
            \hline
            {\textbf{Controlador}} & {$K_P$}                               & {$T_I$}                & {$T_D$}       \\
            \hline
            {\textbf{P}}           & {$\frac{1}{K}\frac{\tau}{\theta}$}    & {$\infty$}             & {$0$}         \\
            \hline
            {\textbf{PI}}          & {$0.9\frac{1}{K}\frac{\tau}{\theta}$} & {$\frac{\theta}{0.3}$} & {$0$}         \\
            \hline
            {\textbf{PID}}         & {$1.2\frac{1}{K}\frac{\tau}{\theta}$} & {$2\theta$}            & {$0.5\theta$} \\
            \hline
        \end{tabular}
        \caption{Método de Ziegler e Nichols para Curva de Reação}
        \label{tab:zncntb}
    \end{center}
\end{table}

Os valores de $T_I$ e $T_D$ são $1/K_i$ e $1/K_d$, respectivamente.

\subsubsection{Cohen Coon}

Criado por Cohen e Coon como uma forma de obter os ganhos de controlador, de forma similar ao método de Ziegler e Nichols
(\ref{subsubsec:znctr}), para modelos clássicos de primeira ordem com atraso.
Utiliza a seguinte tabela para determinar os ganhos de controlador PID:

\begin{table}[h]
    \begin{center}
        \begin{tabular}{ | l | c | c | c | }
            \hline
            {\textbf{Controlador}} & {$K_P$}                               & {$T_I$}                & {$T_D$}       \\
            \hline
            {\textbf{P}}           & {$\frac{1}{K}\frac{\tau}{\theta}[1+\frac{\theta}{3\tau}]$}    & {$\infty$}             & {$0$}         \\
            \hline
            {\textbf{PI}}          & {$\frac{1}{K}\frac{\tau}{\theta}[0.9+\frac{\theta}{12\tau}]$} & {$\frac{\theta [30+3 \frac{\theta}{\tau}]}{9+20 \frac{\theta}{\tau}}$} & {$0$}         \\
            \hline
            {\textbf{PID}}         & {$\frac{1}{K}\frac{\tau}{\theta}[\frac{16\tau+3\theta}{12\tau}]$} & {$\frac{\theta [32+6 \frac{\theta}{\tau}]}{13+8 \frac{\theta}{\tau}}$}            & {$\frac{4 \theta}{11+2 \frac{\theta}{\tau}}$} \\
            \hline
        \end{tabular}
        \caption{Método de Cohen e Coon para Curva de Reação}
        \label{tab:cccntb}
    \end{center}
\end{table}

Os valores de $T_I$ e $T_D$ são $1/K_i$ e $1/K_d$, respectivamente.

\section{Implementação em Python}

Introdução sobre técnicas e ferramentas para implementação de uma lib em python

\subsection{Bibliotecas em Python}

Fundamentação Bibliotecas em Python (cite pip and pypi)

\subsection{Documentação de código}

Fundamentação Documentação de código (sphinx, rtd)

\subsection{Controle em Python}

Fundamentação Controle em Python (control)

\subsection{Outras Bibliotecas}

Fundamentação Outras libs usadas

\subsection{Ferramentas Auxiliares ao Desenvolvimento}



\chapter{Desenvolvimento}
Neste capítulo são apresentados a metodologia e detalhes sobre o desenvolvimento do trabalho.
Nos itens deste capítulo será apresentada a metodologia e detalhado o desenvolvimento do software da biblioteca,
demonstrando requisitos, arquitetura, ambiente de implementação, documentação e implementação.


\section{Metodologia}
Esta seção descreve a metodologia proposta para o desenvolvimento da biblioteca a ser produzida por este trabalho,
levando em consideração as peculiaridades do trabalho e os objetivos específicos a serem alcançados.

A biblioteca será desenvolvido utilizando uma abordagem iterativa incremental, conforme descrito por, \cite{met}
incrementada com as características de metodologia vistas como necessárias para o desenvolvimento de uma biblioteca em
Python.
Esta metodologia permite refinamentos contínuos e adições incrementais de funcionalidades, garantindo que o software
evolua de forma eficaz ao longo do projeto, além de prever etapas de estudo e estruturação, essenciais para esse
tipo de de projeto.
As etapas incluem:

\begin{itemize}
    \item Escolha do escopo do projeto: Alinhamento e definição do que será implementado;
    \item Estudo dos conceitos e ferramentas: Estudo de conceitos e ferramentas que serão utilizados durante o projeto;
    \item Estudo dos casos de uso: Estudo e criação dos casos de uso para compreensão dos requisitos funcionais gerais;
    \item Prototipagem Rápida: Criação de versões iniciais para testar conceitos básicos;
    \item Projeto da Estrutura: Desenvolvimento dos diagramas de classe para estruturar a implementação;
    \item Desenvolvimento Iterativo: Melhoria contínua do software através de ciclos repetidos de desenvolvimento e teste;
    \item Documentação: Desenvolvimento da documentação da biblioteca;
    \item Expansão Incremental: Adição gradual de novas funcionalidades, alinhadas com os requisitos do projeto.
\end{itemize}

Reuniões periódicas com a orientadora serão realizadas para acompanhamento, esclarecimento de dúvidas e ajustes
necessários.

\section{Requisitos da Biblioteca}

O levantamento de requisitos e concepção inicial da dinâmica e uso da biblioteca, foi realizado com base no caminho de
acesso as funcionalidades que um usuário da biblioteca faria, de forma a levar em conta os objetivos específicos
propostos na introdução do trabalho.

O resultado da análise foi o seguinte diagrama de casos de uso, que denota um caminho provável de ações do usuário:

\begin{figure}[H]
    \centering
    \caption{Casos de uso}
    \includegraphics[scale=0.5]{figuras/use_cases}
    \label{fig:use_cases}
    \\
    \vspace{0cm}\hspace{0cm}\small{Fonte: Do autor}
\end{figure}

As tabelas \ref{tab:uc1}, \ref{tab:uc2}, \ref{tab:uc3}, \ref{tab:uc4}, \ref{tab:uc5} detalham os casos de uso ilustrado
na figura \ref{fig:use_cases}, especificando o objetivo, os pré requisitos e o cenário de cada caso de uso.


\begin{table}[!htbp]
    \begin{center}
        \begin{tabularx}{\textwidth}{|>{\bfseries\raggedright\arraybackslash\center}m{5cm}|X|}
            \hline
            Identificador UC: UC-1\newline Diagrama ID: D-1 & Nome: Obtenção de layout de input de dados para identificação\newline Prioridade: Alta                                                                                                                                                                                                   \\ \hline
            Objetivo                                        & Fornecer planilha de layout para que sejam inseridos os dados necessários para um determinado processo de identificação.                                                                                                                                                                 \\ \hline
            Atores                                          & Usuário                                                                                                                                                                                                                                                                                  \\ \hline
            Restrições                                      & Deve-se salvar a planilha referente ao layout necessário.                                                                                                                                                                                                                                \\ \hline
            Pré-condições                                   & -                                                                                                                                                                                                                                                                                        \\ \hline
            Cenário principal                               & 1. O método é chamado, recebendo o caminho onde deve salvar o arquivo e possívelmente os parâmetros referentes ao layout em espessífico, como tipo de arquivo, por exemplo.\newline 2. É construido em memória o layout desejado.\newline 3. É salvo o arquivo no caminho espessificado. \\ \hline
            Pós condições                                   & O Usuário possui o layout e pode inserir os dados.                                                                                                                                                                                                                                       \\ \hline
        \end{tabularx}
        \caption{Obtenção de layout de input de dados para identificação}
        \label{tab:uc1}
    \end{center}
\end{table}

\begin{table}[!htbp]
    \begin{center}
        \begin{tabularx}{\textwidth}{|>{\bfseries\raggedright\arraybackslash\center}m{5cm}|X|}
            \hline
            Identificador UC: UC-2\newline Diagrama ID: D-2 & Nome: Indentificação de sistema\newline Prioridade: Alta                                                                                                                                                                                                                                             \\ \hline
            Objetivo                                        & Realizar processo de identificação do sistema baseado nos dados e obter modelo.                                                                                                                                                                                                                      \\ \hline
            Atores                                          & Usuário                                                                                                                                                                                                                                                                                              \\ \hline
            Restrições                                      & Deve-se obter o modelo do sistema.                                                                                                                                                                                                                                                                   \\ \hline
            Pré-condições                                   & layout de dados preenchido.                                                                                                                                                                                                                                                                          \\ \hline
            Cenário principal                               & 1. O método de identificação é chamado, recebendo o caminho do arquivo de layout e possívelmente os parâmetros referentes ao método em espessífico.\newline 2. São realizados os cálculos e é obtido o objeto de modelo do sistema.\newline 3. O objeto de modelo do sistema é retornado ao usuário. \\ \hline
            Pós condições                                   & O Usuário possui um objeto de modelo do sistema.                                                                                                                                                                                                                                                     \\ \hline
        \end{tabularx}
        \caption{Indentificação de sistema}
        \label{tab:uc2}
    \end{center}
\end{table}

\begin{table}[!htbp]
    \begin{center}
        \begin{tabularx}{\textwidth}{|>{\bfseries\raggedright\arraybackslash\center}m{5cm}|X|}
            \hline
            Identificador UC: UC-3\newline Diagrama ID: D-3 & Nome: Visualização de modelo de sistema\newline Prioridade: Alta                                                                                                                                                                                                                         \\ \hline
            Objetivo                                        & Visualizar gráfico e dados indicadores importantes de um modelo.                                                                                                                                                                                                                         \\ \hline
            Atores                                          & Usuário                                                                                                                                                                                                                                                                                  \\ \hline
            Restrições                                      & Deve ser possível visualisar gráfico e dados indicadores importantes do modelo.                                                                                                                                                                                                          \\ \hline
            Pré-condições                                   & O Usuário possui um objeto de modelo do sistema.                                                                                                                                                                                                                                         \\ \hline
            Cenário principal                               & 1. Um método de visualisação de dados ou gráfico é chamado, passando possíveis parâmetros necessários ou que irão mudar a forma como os dados serão exibidos.\newline 2. São preparados os objetos a serem exibidos ao usuário.\newline 3. São exibidos os dados ou gráficos ao usuário. \\ \hline
            Pós condições                                   & O Usuário pode visualisar os dados do modelo.                                                                                                                                                                                                                                            \\ \hline
        \end{tabularx}
        \caption{Visualização de modelo de sistema}
        \label{tab:uc3}
    \end{center}
\end{table}

\begin{table}[!htbp]
    \begin{center}
        \begin{tabularx}{\textwidth}{|>{\bfseries\raggedright\arraybackslash\center}m{5cm}|X|}
            \hline
            Identificador UC: UC-4\newline Diagrama ID: D-4 & Nome: Aproximação de Controlador baseado em modelo de sistema\newline Prioridade: Alta                                                                                                                                                                                                     \\ \hline
            Objetivo                                        & Obter aproximação de parâmetros de controle baseados no modelo obtido.                                                                                                                                                                                                                     \\ \hline
            Atores                                          & Usuário                                                                                                                                                                                                                                                                                    \\ \hline
            Restrições                                      & Deve-se obter uma aproximação dos parâmetros do controlador.                                                                                                                                                                                                                               \\ \hline
            Pré-condições                                   & O Usuário possui um objeto de modelo do sistema.                                                                                                                                                                                                                                           \\ \hline
            Cenário principal                               & 1. O método de aproximação é chamado, recebendo o modelo do sistema e possívelmente os parâmetros referentes ao método de aproximação em espessífico.\newline 2. São realizados os cálculos e é obtido o objeto de controlador.\newline 3. O objeto de controlador é retornado ao usuário. \\ \hline
            Pós condições                                   & O Usuário possui um objeto de controlador.                                                                                                                                                                                                                                                 \\ \hline
        \end{tabularx}
        \caption{Aproximação de Controlador baseado em modelo de sistema}
        \label{tab:uc4}
    \end{center}
\end{table}

\begin{table}[!htbp]
    \begin{center}
        \begin{tabularx}{\textwidth}{|>{\bfseries\raggedright\arraybackslash\center}m{5cm}|X|}
            \hline
            Identificador UC: UC-5\newline Diagrama ID: D-5 & Nome: Visualização de controlador\newline Prioridade: Alta                                                                                                                                                                                                                               \\ \hline
            Objetivo                                        & Visualizar gráfico e dados indicadores importantes de um controlador.                                                                                                                                                                                                                    \\ \hline
            Atores                                          & Usuário                                                                                                                                                                                                                                                                                  \\ \hline
            Restrições                                      & Deve ser possível visualisar gráfico e dados indicadores importantes do controlador gerado.                                                                                                                                                                                              \\ \hline
            Pré-condições                                   & O Usuário possui um objeto de controlador.                                                                                                                                                                                                                                               \\ \hline
            Cenário principal                               & 1. Um método de visualisação de dados ou gráfico é chamado, passando possíveis parâmetros necessários ou que irão mudar a forma como os dados serão exibidos.\newline 2. São preparados os objetos a serem exibidos ao usuário.\newline 3. São exibidos os dados ou gráficos ao usuário. \\ \hline
            Pós condições                                   & O Usuário pode visualisar os dados do controlador.                                                                                                                                                                                                                                       \\ \hline
        \end{tabularx}
        \caption{Visualização de controlador}
        \label{tab:uc5}
    \end{center}
\end{table}




\section{Descrição da arquitetura}\label{sec:descarc}

A fim de construir uma biblioteca com arquitetura de código compreensível e escalável, optou-se pela orientação
da mesma a classes e objetos.
Baseado nisso e nos casos de uso, foi desenvolvido o diagrama de classes da figura \ref{fig:class_diag}.
Onde foram definidas as seguintes classes:
\begin{alineas}
    \item \textbf{DataInputUtils}: Classe utilitária para a entrada de dados externos - pode ser melhor visualizada na figura \ref{fig:class_diag_diubmi};
    \item \textbf{DataUtils}: Classe utilitária para manipulação de dados em memória - pode ser melhor visualizada na figura \ref{fig:class_diag_dupu};
    \item \textbf{PlotUtils}: Classe utilitária para plot de funções de transferência e outras informações relacionadas - pode ser melhor visualizada na figura \ref{fig:class_diag_dupu};
    \item \textbf{Model}: Classe representativa do Modelo matemático de uma planta de sistemas de controle - pode ser melhor visualizada na figura \ref{fig:class_diag_model};
    \item \textbf{ModelView}: Classe utilizada pra visualização de dados de um objeto da classe Model - pode ser melhor visualizada na figura \ref{fig:class_diag_model};
    \item \textbf{BaseModelIdentification}: Classe utilitária base para Identificação de modelos (Model), herdada por classes que implementam métodos de identificação - pode ser melhor visualizada na figura \ref{fig:class_diag_diubmi};
    \item \textbf{Controller}: Classe representativa do Modelo matemático de uma planta de sistemas de controle em Malha Fechada com Controlador PID - pode ser melhor visualizada na figura \ref{fig:class_diag_controller};
    \item \textbf{ControllerView}: Classe utilizada pra visualização de dados de um objeto da classe Controller - pode ser melhor visualizada na figura \ref{fig:class_diag_controller};
    \item \textbf{BaseControllerAproximation}: Classe utilitária base para aproximação de controladores (Controller), herdada por classes que implementam métodos de aproximação de parâmetros de controlador PID - pode ser melhor visualizada na figura \ref{fig:class_diag_bcacontroller}.
\end{alineas}

\begin{figure}[H]
    \centering
    \caption{Diagrama de classes}
    \includegraphics[scale=0.32]{figuras/class_diag}
    \label{fig:class_diag}
    \\
    \vspace{0cm}\hspace{0cm}\small{Fonte: Do autor}
\end{figure}

\begin{figure}[H]
    \centering
    \caption{Diagrama de classes ampliado - DataInputUtils e BaseModelIdentification}
    \includegraphics[scale=0.6]{figuras/class_diag_diubmi}
    \label{fig:class_diag_diubmi}
    \\
    \vspace{0cm}\hspace{0cm}\small{Fonte: Do autor}
\end{figure}

\begin{figure}[H]
    \centering
    \caption{Diagrama de classes ampliado - DataUtils e PlotUtils}
    \includegraphics[scale=0.6]{figuras/class_diag_dupu}
    \label{fig:class_diag_dupu}
    \\
    \vspace{0cm}\hspace{0cm}\small{Fonte: Do autor}
\end{figure}

\begin{figure}[H]
    \centering
    \caption{Diagrama de classes ampliado - Model e ModelView}
    \includegraphics[scale=0.7]{figuras/class_diag_model}
    \label{fig:class_diag_model}
    \\
    \vspace{0cm}\hspace{0cm}\small{Fonte: Do autor}
\end{figure}

\begin{figure}[H]
    \centering
    \caption{Diagrama de classes ampliado - Controller e ControllerView}
    \includegraphics[scale=0.7]{figuras/class_diag_controller}
    \label{fig:class_diag_controller}
    \\
    \vspace{0cm}\hspace{0cm}\small{Fonte: Do autor}
\end{figure}

\begin{figure}[H]
    \centering
    \caption{Diagrama de classes ampliado - BaseControllerAproximation}
    \includegraphics[scale=0.6]{figuras/class_diag_bcacontroller}
    \label{fig:class_diag_bcacontroller}
    \\
    \vspace{0cm}\hspace{0cm}\small{Fonte: Do autor}
\end{figure}

\section{Descrição do ambiente de implementação}

Nesta sessão serão explorados itens referentes ao ambiente de implementação utilizado para o
desenvolvimento, todos visando necessidades para implementação e publicação da biblioteca ou boas práticas de
desenvolvimento.

Para possibilitar a manutenção do código e facilitar futuras implementações é necessária a adoção de uma
estrutura de código e pastas que seja comum a comunidade, além disso existe a necessidade de uma estrutura compatível
com a publicação do código da biblioteca e do isolamento do código fonte da biblioteca dos testes da mesma.
Com isso em mente, foi adotada a estrutura apresentada em \cite{auto_test_vid}, que atende a essas necessidades.

\subsection{Controle de versão}

Foi escolhido o Git como opção de sistema de controle de versões da biblioteca.
Ele é um sistema de controle de versões preferido da comunidade, devido a sua alta performance, natureza decentralizada
e funcionalidades robustas para manipulação de grandes projetos \cite{usegit}.
Além de ser gratuito e de código aberto, é amplamente utilizado e integrado em um pletora de ferramentas e repositórios
como GitHub.

\subsection{Verificações e Testes}
\begin{figure}
    \centering
    \includegraphics{C:\Users\luiz\PycharmProjects\tcc\figuras\get_model_plot}
    \caption{}
    \label{fig:}
\end{figure}
Nesta sessão será abordado o uso de ferramentas de verificação de código como Mypy e Flake8, bem como ferramentas
que foram utilizadas para automação de testes, como Pytest, tox e GitHub actions.

\subsubsection{Mypy e Flake8}
Ambas as ferramentas Mypy e Flake8 foram utilizadas durante o desenvolvimento para verificar erros de tipagem e
adequação ao PEP 8, de forma que o resultado final passa nas verificações de ambas as ferramentas.
Elas também foram adicionadas ao processo de automação de testes descrito em \ref{subsubsec:devtox}, para que não
ficassem apenas a cardo do uso manual pelo desenvolvedor.

\subsubsection{Pytest}

Para cada uma das classes implementadas foram desenvolvidos testes unitários, para validação de suas funcionalidades e
garantia de que implementações futuras que alterem os resultados de funcionalidades já implementadas, possivelmente de
forma incorreta, não passem despercebidas gerando falhas ao rodar os testes.

\subsubsection{tox}\label{subsubsec:devtox}

A ferramenta tox foi utilizada para automatizar a execução das verificações, com Mypy e Flake8, e testes unitários, com
PyTest, para as versões 3.9 e 3.10 do Python.
De forma que ao roda-lo, com base no arquivo de configuração criado, são instaladas as dependências do projeto em um
ambiente virtual e são executadas todas as verificações e testes.

\subsubsection{GitHub Actions}

Foi utilizada a plataforma de CI/CD do GitHub, GitHub Actions para a criação de um workflow de testes, que roda a
ferramenta tox descrita em \ref{subsubsec:devtox}, em ambientes Windows e Linux, toda vez que um pull é feito em uma
branch ou que é realizado um merge request.
Desta forma é automatizado o processo de testagem em diversos ambientes, que pode ser mais demorado e atrapalhar o
fluxo de desenvolvimento.
Além de garantir que os testes sejam sempre rodados para toda alteração que for feita no repositório.

\subsection{Documentação do código}

A documentação de código foi desenvolvida juntamente as implementações de cada classe.
Optou-se pelo uso da ferramenta Sphinx para a documentação de código utilizando o tema do projeto Read the Docs, onde
também foi hospedada como um projeto de código aberto e seguindo uma estrutura de documentação similar a da
biblioteca Python Control Systems Library.

\subsubsection{Sphinx}

Para a documentação em Sphinx foram desenvolvidas diversas páginas em reStructuredText tratando dos seguintes itens:
\begin{alineas}
    \item \textbf{index}: Página principal da documentação com breve introdução e sumário da documentação;
    \item \textbf{Sobre}: Visão geral do projeto e guia de instalação;
    \item \textbf{Introdução}: Início rápido com exemplos e explicações práticas, e explicação geral do funcionamento da biblioteca;
    \item \textbf{Referência de Classes}: Listagem de todas as classes implementadas com breve resumo para cada grupo de classes;
    \item \textbf{Referências}: Página de glossário e referências bibliográficas utilizadas na documentação;
    \item \textbf{Desenvolvimento}: Página com orientações ao desenvolvimento.
\end{alineas}

Além disso a documentação de todas as classes e métodos implementados foi feita juntamente ao código fonte, através de
docstrings suportadas que foram interpretadas pelo Sphinx, que gerou páginas da documentação para cada classe, e essas
ficaram facilmente acessíveis através da página de referência de classes.

Algumas extensões foram adicionadas ao projeto para facilitar a documentação e melhorar a visualização e usabilidade:

\begin{alineas}
    \item \textbf{autodoc}: Utilizada para gerar documentação automaticamente baseado nas docstrings;
    \item \textbf{intersphinx}: Possibilita links entre documentações de diferentes bibliotecas;
    \item \textbf{mathjax}: Suporte a expressões matemáticas no formato LaTex;
    \item \textbf{autosummary}: Utilizada para gerar sumários de documentação automaticamente baseado nas docstrings;
    \item \textbf{napoleon}: Facilidade de sintaxe nas docstrings;
    \item \textbf{numpydoc}: Facilidade de sintaxe nas docstrings;
    \item \textbf{linkcode}: Adiciona link direto da documentação para o código fonte no repositório;
    \item \textbf{sphinx\_paramlinks}: Possibilita referencias parâmetros nas docstrings;
    \item \textbf{bibtex}: Suporte a referências do tipo BibTex;
    \item \textbf{sphinx\_rtd\_dark\_mode}: Modo escuro para o tema Read the Docs.
\end{alineas}

\subsubsection{Read the Docs}

Foi optado pelo uso do tema do projeto Read the Docs por questões gostos visuais e facilidade de navegação pela
documentação.
Além disso, foi utilizada a hospedagem gratuita para projetos de código aberto e da comunidade fornecida pela Read the
Docs.
Uma integração com o repositório git é fornecida para que toda vez que a branch principal for atualizada no repositório
a documentação no Read the Docs também seja atualizada automaticamente.

A documentação de código pode ser acessada por qualquer um na íntegra pelo link disponível no apêndice \ref{ch:actdocs}.

\subsection{Disponibilização do código}

Nesta sessão são descritas as formas como o código foi disponibilizado para uso, leitura e sugestões de melhoria.

\subsubsection{GitHub}

Visando disponibilizar o código fonte como código aberto e acessível, foi criado um repositório público e gratuito na
plataforma GitHub, cujo link pode ser encontrado no apêndice \ref{ch:actgithub}, onde qualquer um pode ter acesso a todo
o código fonte desenvolvido, bem como apontar bugs e sugerir alterações.

\subsubsection{PyPI}

Após a conclusão do desenvolvimento da biblioteca, iniciou-se o processo de disponibilização no PyPI\@.
Para isso foram criados arquivos de configuração do projeto, com detalhes como nome, versão, autor e requerimentos
para instalação.
Por fim o pacote foi carregado o PyPI utilizando a ferramenta twine, garantindo um upload seguro.
Este processo tornou o a biblioteca acessível para instalação via pip para qualquer pessoa.

\section{Implementação}
Esta seção detalha a implementação do código-fonte principal da biblioteca, contendo todas as funcionalidades
disponibilizadas a usuários dela.
Da mesma forma que na documentação da biblioteca as subseções desta seção representam grupos de classes implementadas
baseando-se nos diagramas apresentados na seção \ref{sec:descarc}.

\subsection{Modelo}

Classes de modelo, representativas do Modelo matemático de uma planta de sistemas de controle.

\subsubsection{Model}

Classe representativa do Modelo matemático de uma planta de sistemas de controle.
Tipicamente o modelo matemático de uma planta no domínio da frequenica pode ser definido por um numerador e um
denominador em potências de $s$, como, por exemplo:
\begin{equation}
    \label{eq:modelex}
    P(s) = \frac{ s + 1 }{ s^2 + s + 1 }
\end{equation}

Esta classe funciona guardando um objeto de Função de Transferência (tf) da biblioteca de sistemas de controle do Python
(control), representando o modelo matemático de uma planta no domínio da frequenica, além de guardar alguns outros
metadados sobre a função de transferência, que poderão ser utilizados para aproximação de controlador PID
posteriormente, por exemplo.

Além disso o atributo view, um objeto da classe ModelView possibilita a visualização de dados, estatísticas e gráficos
referentes a função de transferência.

Para ser instanciada, recebe um parâmetro tf, referente a função de transferência, e um parâmetro opcional source\_data
referente aos dados utilizados como base para gerar o modelo, caso o modelo tenha sido gerado dessa forma.

\subsubsection{ModelView}

Classe utilizada pra visualização de dados de um objeto da classe Model.
Recebe Model como parâmetro e tem como foco a visualização dos dados do mesmo.
Para as apresentações visuais, faz uso dos métodos explorados em \ref{subsec:dataviz}, e implementa os seguintes
métodos para possibilitar essa visualização:
\begin{alineas}
    \item \textbf{plot\_model\_step\_response\_graph}: Realiza a plotagem do gráfico da resposta a sinal degrau do
    Modelo, bem como as retas de tempo de acomodação, sobressinal, e valor de regime e os dados discretos caso tenham
    sido informados;
    \item \textbf{get\_model\_step\_response\_data}: Utiliza control.step\_info da biblioteca de controle para
    obtenção dos dados de resposta a sinal degrau do sistema no formato de dicionário do Python;
    \item \textbf{print\_model\_step\_response\_data}: Imprime em tela os dados de resposta a sinal degrau do sistema;
    \item \textbf{print\_tf}: Imprime em tela a função de transferência do modelo, com formatação matemática caso
    esteja sendo executado em ambiente Jupyter.
\end{alineas}

\subsubsection{FirstOrderModel}\label{subsubsec:fom}

Essa classe é voltada um Modelo paramétrico da dinâmica de um processo comumente encontrado na indústria,
caracterizado pela função de transferência \eqref{eq:firstordertf} detalhado em \ref{subsec:modelfund}. Sendo uma
subclasse de Model, essa classe adiciona suporte a definição de um modelo com os parâmetros K ($K$), tau
($\tau$) e theta ($\theta$) mas ainda mantém todas as funcionalidades da classe pai. Adicionalmente, ela faz uso do
atributo pade, caso $\theta$ seja diferente de zero.
Nele é salva a aproximação de padê para o atraso do sistema.

\subsection{Métodos de Identificação}

Classes de identificação são subclasses de BaseModelIdentification, e implementam o método get\_model, que faz a
identificação de dados discretos de resposta a sinal degrau de um sistema de controle e retorna objeto da classe Model
ou de uma de suas subclasses.

\subsubsection{BaseModelIdentification}
Classe abstrata que serve de base para identificação de modelos (Model).
Classes de métodos de identificação de modelos devem ser subclasses desta classe.
Elas também devem implementar o método get\_model para retornar um objeto da classe Model que represente o modelo
matemático do sistema que produziu os dados recebidos.

Em geral, implementações farão a Identificação com base nos dados de resposta a sinal degrau.
Para tanto, o método get\_data\_input\_layout oferece um leiaute para serem informados os dados referentes a resposta
do sistema a um sinal degrau em relação ao tempo.
Implementações de get\_model podem ler os dados e aplicar seus métodos de identificação específicos.

\subsubsection{ZieglerNicholsModelIdentification}

Como uma uma subclasse de BaseModelIdentification, implementa o método de identificação de Ziegler e Nichols,
descrito em \ref{subsubsec:znfun}.
Ela sobreescreve o método abstrato da classe pai get\_model, implementando para que, com base nos dados de resposta a
sinal degrau fornecidos e em outros parâmetros opcionais, sejam calculados os parâmetros para instanciar um objeto da
classe FirstOrderModel:

\begin{alineas}
    \item Utiliza DataInputUtils.get\_model\_data\_default para obter a tabela (pandas.DataFrame) com os dados de
    resposta do modelo;
    \item Então obtém a pandas.Series tf\_data (lista cujo índice representa o tempo e os valores a saida)
    e o valor do sinal degrau através do método DataUtils.setup\_data\_default;
    \item Obtém o valor de regime e o momento de entrada em valor de regime através do método
    DataUtils.get\_vreg;
    \item Obtém o valor da inclinação da reta tangente e o momento em que a reta encosta na curva $y(t)$
    através do método DataUtils.get\_max\_tan;
    \item Encontra o valor de $y(t)$ quando $t$ é igual ao momento em que a reta encosta na curva $y(t)$;
    \item Utiliza a inclinação da reta tangente e a localização do ponto de encontro dela com a curva $y(t)$
    para encontrar os valores de $t_1$ e $t_3$ através do método
    DataUtils.get\_time\_from\_inclination e dos valores de $y(t)$ referentes aos pontos desejados,
    $y(t) = 0$ e $y(t) = y_f$, respectivamente;
    \item Obtém $K$ dividindo o valor de regime, $y_f$, pelo valor do sinal degrau;
    \item Com os valores de $t_1$ e $t_3$ em mãos calcula o valor de tau e theta,
    sendo $\tau = t_3 - t_1$ e $\theta = t_1$`;
    \item Dependendo do valor de theta e do parâmetro ignore\_delay\_threshold zera o valor de theta;
    \item Instancia um objeto da classe FirstOrderModel com os valores obtidos.
\end{alineas}

\subsubsection{HagglundModelIdentification}

Muito similar a ZieglerNicholsModelIdentification a classe HagglundModelIdentification implementa o método de
identificação de Hägglund descrito em \ref{subsubsec:hagfun}.
Com a única diferença sendo a determinação de $t2$ baseado em $y_f*0.632$ e o calculo de $\theta$ baseado em $t_1$ e
$t_2$.

\subsubsection{SmithModelIdentification}

Seguindo o que foi descrito em \ref{subsubsec:smfun} a classe SmithModelIdentification implementa o método get\_model da
segunte forma:

\begin{alineas}
    \item Utiliza DataInputUtils.get\_model\_data\_default para obter a tabela (pandas.DataFrame) com os dados de resposta do modelo;
    \item Então obtém a pandas.Series tf\_data (lista cujo index representa o tempo e os valores a saida) e o valor do sinal degrau através do método DataUtils.setup\_data\_default;
    \item Obtém o valor de regime e o momento de entrada em valor de regime através do método DataUtils.get\_vreg;
    \item Calcula $t_1$ e $t_2$ baseado no instante em que a curva atinge 28.3\% e 63.2\% do valor de regime, respectivamente;
    \item Obtém $K$ dividindo o valor de regime, pelo valor do sinal degrau;
    \item Com os valores de $t_1$ e $t_2$ em mãos, calcula o valor de tau e theta, sendo $\tau = 1.5*(t_2 - t_1)$ e $\theta = t_2 - \tau$;
    \item Dependendo do valor de theta e de ignore\_delay\_threshold zera o valor de theta;
    \item Instancia um objeto da classe FirstOrderModel com os valores obtidos.
\end{alineas}

\subsubsection{SundaresanKrishnaswamyModelIdentification}

Implementado de forma idêntica a SmithModelIdentification e com base na seção \ref{subsubsec:skfun}, as únicas
diferenças na implementação são o cálculo de $t_1$ e $t_2$ baseado no instante em que a curva atinge 35.3\% e 85.3\% do
valor de regime, respectivamente, e o cálculo do valor de tau e theta, sendo $\tau = 0.67*(t_2 - t_1)$ e
$\theta = 1.3t_1 - 0.29t_2$.

\subsubsection{NishikawaModelIdentification}
A implementação de get\_model de NishikawaModelIdentification funciona da seguinte forma:

\begin{alineas}
    \item Utiliza DataInputUtils.get\_model\_data\_default para obter a tabela (pandas.DataFrame) com os dados de resposta do modelo.
    \item Então obtém a pandas.Series tf\_data (lista cujo index representa o tempo e os valores a saida) e o valor do sinal degrau através do método DataUtils.setup\_data\_default.
    \item Obtém o valor de regime e o momento de entrada em valor de regime através do método DataUtils.get\_vreg.
    \item Calcula $A_0$ como a área de um retângulo de lados $\Delta y(\infty)$ e o momento de entrada em valor de regime menos a integral da curva de $t = 0$ até momento de entrada em valor de regime.
    \item Calcula $t_0$ como $A_0$ dividido pelo valor de regime.
    \item Calcula $A_1$ como a integral da curva de $t = 0$ até $t = t_0$.
    \item Obtém $K$ dividindo o valor de regime, pelo valor do sinal degrau.
    \item Com os valores de $A_0$, $A_1$ e $t_0$ em mãos, calcula o valor de tau e theta, sendo $\tau$ igual a $A_0$ dividido por 0.368 vezes o valor de regime e $\theta = t_0 - \tau$.
    \item Dependendo do valor de theta e de ignore\_delay\_threshold zera o valor de theta.
    \item Instancia um objeto da classe FirstOrderModel com os valores obtidos.
\end{alineas}

\subsection{Controlador}

Classes de controlador, representativas de um sistema em Malha Fechada ocorrendo o controle de um Modelo através de um
Controlador PID\@.

\subsubsection{Controller}

Classe base para controle PID\@.
Esta é uma classe representativa do Modelo matemático de uma planta de sistemas de controle em Malha Fechada com
Controlador PID\@.
Ela se destina a armazenar os valores de ganhos de controlador, bem como realizar o fechamento da manha de controle
utilizando os parâmetros PID (kp, ki, kd) e o objeto de modelo (model).
Para isso, atua encima do objeto de Função de Transferência da biblioteca de controle Model.tf
(control.TransferFunction) presente no Model fornecido.

Além disso o atributo view possibilita a visualização de dados, estatísticas e gráficos referentes ao resultado do
controle (função de transferência resultante do fechamento da malha com controlador).

\subsubsection{ControllerView}

Classe utilizada pra visualização de dados de um objeto da classe Controller.
Recebe Controller como parâmetro e tem como foco a visualização dos dados do mesmo.
Para as apresentações visuais, faz uso dos métodos explorados em \ref{subsec:dataviz}, e implementa os seguintes
métodos para possibilitar essa visualização:
\begin{alineas}
    \item \textbf{plot\_model\_step\_response\_graph}: Realiza a plotagem do gráfico da resposta a sinal degrau do
    modelo do sistema em malha fechada com o controlador, bem como as retas de tempo de acomodação, sobressinal, e valor
    de regime.
    Também realiza a plotagem da da resposta a sinal degrau de Model, para comparação por padrão;
    \item \textbf{get\_model\_step\_response\_data}: Utiliza control.step\_info da biblioteca de controle para
    obtenção dos dados de resposta a sinal degrau do sistema no formato de dicionário do Python;
    \item \textbf{print\_model\_step\_response\_data}: Imprime em tela os dados de resposta a sinal degrau do sistema;
    \item \textbf{print\_tf}: Imprime em tela a função de transferência do modelo, com formatação matemática caso
    esteja sendo executado em ambiente Jupyter.
\end{alineas}

\subsection{Métodos De Aproximação de Controlador}

Classes de Aproximação de Controlador são subclasses de BaseControllerAproximation, e implementam o método
get\_controller, que faz a Aproximação de Ganhos de um Controlador PID para um objeto de Modelo da classe Model ou
uma de suas subclasses, retornando um objeto da classe Controller.

\subsubsection{BaseControllerAproximation}

Classe base para aproximação de controladores (Controller).
Métodos de aproximação de controladores devem ser subclasses desta classe.
Elas também devem implementar o método get\_controller para retornar um objeto da classe Controller com os ganhos
referentes de Controlador PI\@.
Subclasses podem sobrescrever o atributo \_accepted\_controllers com os tipos de controlador suportados e utilizar o
método \_parse\_controller\_option() para verificar se um tipo de controlador é aceito.

\subsubsection{Aproximação de ganhos de controlador para modelos de primeira ordem por tabela}\label{subsubsec:agcmpot}

Na literatura existem diversos métodos de aproximação de ganhos de controlador para modelos de primeira ordem
como o descrito na equação \ref{eq:firstordertf}, que possuem tabelas de fórmulas para o cálculo dos ganhos de
controlador PID, que variam de acordo com o tipo de controlador e fazem uso exclusivo dos parâmetros $K$, $\tau$ e
$\theta$.
Com isso em mente foi desenvolvida a classe FirstOrderTableControllerAproximation, uma subclasse de
BaseControllerAproximation especializada para esses casos.
A classe visa facilitar a implementação destes comumente referidos como “Métodos de Tabela”, pois possuem formulas
simples e específicas para o ganho de cada parâmetro de PID, a depender do controlador desejado.

Classes de aproximação de controlador PID que se encaixarem podem ser implementadas como subclasses de
FirstOrderTableControllerAproximation, para isso, devem ser seguidos os seguintes passos:

\begin{alineas}
    \item Adicionar a classe FirstOrderTableControllerAproximation a herança da nova classe;
    \item Declarar o \_accepted\_controllers com os tipos de controladores aceitos;
    \item Declarar o dicionário \_controller\_table com os tipos de controladores como chaves e referências a subclasses de FirstOrderTableControllerAproximationItem como valores;
    \item Por fim, basta implementar cada uma das classes derivadas de FirstOrderTableControllerAproximationItem.
\end{alineas}

O método get\_controller da classe FirstOrderTableControllerAproximation se encarrega de receber os dados, verificar
se o tipo de controlador é permitido e chamar o método get\_controller do FirstOrderTableControllerAproximationItem
adequado.

Para cada tipo de controlador suportado uma subclasse de FirstOrderTableControllerAproximationItem deve ser criada
implementando o método abstrato get\_controller, que recebe um objeto FirstOrderModel e deve retornar um objeto de
Controlador.

\subsubsubsection{ZieglerNicholsControllerAproximation}

Conforme descrito na seção \ref{subsubsec:znctr} o método de aproximação de controlador Ziegler Nichols se encaixa no
caso explorado pela classe FirstOrderTableControllerAproximation, desta forma foi realizada a implementação conforme
descrito em \ref{subsubsec:agcmpot} para os controladores P, PI e PID\@.

\subsubsubsection{CohenCoonControllerAproximation}

O método de aproximação de ganhos de controlador PID de Cohen e Coon também se encaixa em
FirstOrderTableControllerAproximation, conforme pode ser visto em \ref{subsubsec:ccapx} e de maneira similar a
implementação do método de aproximação de controlador Ziegler Nichols foi implementado como uma subclasse do caso
especializado.

\subsection{Visualização de Dados}\label{subsec:dataviz}

A fim de concentrar ferramentas comuns de visualização de dados, foi criada a classe utilitária PlotUtils, utilizada
tanto nas implementações das classes ModelView como Controller View.
Desta forma não há a necessidade de repetir o código de funcionalidades comuns entre seções diferentes da biblioteca.

\subsubsection{PlotUtils}
Classe utilitária para plotagem de funções de transferência e outras informações relacionadas.
Nela foram implementados os métodos plot\_tf, print\_tf e pprint\_dict, detalhados a seguir.

\subsubsubsection{plot\_tf}
Com base em uma ou mais funções de transferência informadas, realiza a plotagem da resposta a sinal degrau, das retas de
do tempo de acomodação, do sobressinal, e do valor de regime para cada uma.
Também realiza o plot dos dados discretos caso sejam informados, e uma legenda detalhando cada item plotado é
adicionada.

Reconhece caso esteja sendo executado em ambiente Jupyter e configura a biblioteca matplotlib para que o gráfico
seja plotado de forma adequada para o ambiente.
Do contrário configura para que seja aberta uma janela interativa onde o gráfico pode ser explorado.

Algumas opções são fornecidas para a plotagem dos dados como parâmetros opcionais:
\begin{alineas}
    \item \textbf{settling\_time\_threshold}: Limiar de tempo de acomodação, por padrão 0.02 (2\%);
    \item \textbf{pade}: Função de transferência de pade para consideração do atraso na resposta ao sinal degrau;

    \item \textbf{scale}: Fator de escala para as funções de transferência, por padrão 1.
    Se for informado valor, o fator será aplicado a todas as funções de transferência.
    Se um dicionário for informado, o fator será aplicado para as funções de transferência cujo sufixo coincidir com
    alguma chave do dicionário;

    \item \textbf{simulation\_time}: Tempo de simulação, caso não seja informado, o tempo de acomodação da função de
    transferência será usado como base, se não for possível calcular, será utilizado simulation\_time = 100 no lugar;

    \item \textbf{qt\_points}: Quantidade de pontos para a simulação, por padrão 1000.
\end{alineas}

\subsubsubsection{print\_tf}
Imprime uma função de transferência em tela.
Reconhece caso esteja sendo executado em ambiente Jupyter e chama a função display do ambiente para imprimir a função
na tela, de forma que a função é apresentada de maneira formatada como expressão matemática.

\subsubsubsection{pprint\_dict}
Imprime um dicionário de maneira formatada.
Reconhece caso esteja sendo executado em ambiente Jupyter e, se estiver, instancia um pandas.DataFrame com as
informações do dicionário e chama a função display do ambiente para imprimir os dados em tela com um visual mais
agradável.

\subsection{Manipulação de Dados}

A fim de concentrar ferramentas comuns de manipulação de dados, foram criadas as classes utilitárias DataInputUtils e
DataUtils utilizadas em quase todos os módulos da biblioteca.

\subsubsection{DataInputUtils}
Classe utilitária para facilitar a entrada de dados, nela são implementados diversos métodos para criação e leitura de
tabelas referentes a entrada dos dados utilizados para identificação de plantas de sistemas de controle.
Foram implementados os seguintes métodos:

\begin{alineas}
    \item \textbf{create\_table\_with\_fields}: Cria um arquivo (CSV ou Excel) contendo uma tabela vazia com os campos
    especificados;
    \item \textbf{read\_table\_with\_fields}: Lê um arquivo contendo uma tabela e retorna um
    pandas.DataFrame com **apenas** os campos especificados;
    \item \textbf{expected\_fields}: Retorna a lista standard\_fields com exceção dos campos sample\_time ou
    step\_signal caso sejam informados nos parâmetros;
    \item \textbf{get\_model\_data\_default}: Salva planilha de leiaute com os campos esperados para preenchimento.
\end{alineas}

\subsubsection{DataUtils}

Classe utilitária para manipulação de dados, nela foram implementados diversos métodos para manipulação de dados e
cálculo de constantes, como pose ser visto a seguir.


\begin{alineas}
    \item \textbf{linfilter}: Aplica um filtro linear à série temporal de entrada para suavização de dados ruidosos.
    A intensidade da suavisação depende do parâmetro smoothness;
    \item \textbf{get\_vreg}: Obtém o valor de regime, dentro de settling\_time\_threshold, de uma resposta a sinal
    degrau.
    Obtido iterando pelos dados e ignorando valores iniciais até que todos os pontos estejam dentro do desvio aceitado
    da média, o valor do momento mais antigo que sobrou é tomado como valor de regime;
    \item \textbf{get\_max\_tan}: Obtém as coordenadas e o valor da inclinação do ponto de maior inclinação de uma série
    temporal.
    Internamente faz uso do método diff da pandas.Series para obter a série diferencial;
    \item \textbf{get\_time\_at\_value}: Encontra o tempo para um valor de uma reta informada.
    Utilizado pelos métodos de identificação Ziegles Nichols e Hägglund para obtenção dos tempos $t_1$, $t_2$ e $t_3$;
    \item \textbf{setup\_data\_default}: Método padrão para preparo dos dados para identificação de modelo.
    Ele faz o preparo dos dados de adicionando adicionando tempo de aquisição e sinal degrau, caso necessário,
    e aplicando um filtro linear, caso solicitado.
    Retorna a pandas.Series referente a série temporal do sinal de saida em relação ao tempo e o valor do sinal degrau;
    \item \textbf{trunk\_data\_input}: Remove dados onde o valor de entrada é nulo;
    \item \textbf{offset\_data\_output}: Remove valores negativos e interpola os dados para encostarem no eixo do tempo.
\end{alineas}


\chapter{Apresentação dos Resultados}

Quick summary of how results will be compared and verified

\section{Arquitetura}
arquitetura geral atualizada
arquitetura das classes de itentificação e aproximação?

\section{Documentação}

link to docs

\section{Testes}

coverage

\section{Usabilidade}

usability

\section{Comparação com a Bibliografia}

Compare results to bibliography results

\section{comaprações com Experimentos Anteriores}

Compare results to previous experiments

\section{Aplicação em Sistema Real}

Use on a real system

\chapter{Considerações Finais}

final considerations (toptop)

\section{Sugestões para trabalhos futuros}

a lot of stuff hahhahah

integration with control has root locus

more id methods
1
2
3

model aprox methods
1
2
3


higher order work

plot response on other kinds of inputs

auto pypi




% ----------------------------------------------------------
% ELEMENTOS PÓS-TEXTUAIS
% ----------------------------------------------------------
\postextual
% ----------------------------------------------------------

% ----------------------------------------------------------
% Referências bibliográficas
% ----------------------------------------------------------
\bibliography{referencias}

% ----------------------------------------------------------
% Apêndices
% ----------------------------------------------------------
 \begin{apendicesenv}
     \partapendices
     \chapter{Código-fonte}
\label{ch:actgithub}
\url{https://github.com/luizn22/auto-control-tools}

\chapter{Documentação Auto Control Tools}
\label{ch:actdocs}
\url{https://auto-control-tools.readthedocs.io/}

\chapter{Diagrama de Classes Atualizado}
\label{ch:nwclassdiag}

\begin{figure}[H]
    \centering
    \caption{Novo diagrama de classes}
    \includegraphics[scale=0.23, angle=90]{figuras/class_diag_new}
    \label{fig:class_diag_new}
    \\
    \vspace{0cm}\hspace{0cm}\small{Fonte: Do autor}
\end{figure}


 \end{apendicesenv}

% ----------------------------------------------------------
% Anexos
% ----------------------------------------------------------
 \begin{anexosenv}
     \partanexos
     \include{tex/Anexos}
 \end{anexosenv}

%---------------------------------------------------------------------
% INDICE REMISSIVO
%---------------------------------------------------------------------
%\phantompart
%\printindex
%---------------------------------------------------------------------

\end{document}
