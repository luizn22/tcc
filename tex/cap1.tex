\chapter{Introdução}

\abreviatura{PID}{Proporcional-Integral-Derivativo}

O desenvolvimento tecnológico na área de controle de sistemas tem experimentado um crescimento significativo nas
últimas décadas.
Os controladores Proporcional-Integral-Derivativo (PID) são utilizados na maioria das aplicações de controle de
processos automáticos na indústria atualmente, regulando fluxo, temperatura, pressão, nível e muitas outras variáveis
de processos industriais.
Esses controladores, que datam de 1939, são a base dos sistemas modernos de controle de processos, automatizando tarefas
de regulação que, de outra forma, teriam que ser feitas manualmente \cite{introart1}.

Apesar de sua ampla utilização, a configuração eficaz de um controlador PID permanece um desafio, especialmente em
sistemas com dinâmicas complexas e variáveis.
A sintonização desses controladores exige um entendimento profundo do sistema a ser controlado e habilidades
especializadas para ajustar os parâmetros adequadamente \cite{introart2}.

A implementação de controladores PID, embora bem estabelecida, envolve processos que podem ser complexos e demorados,
especialmente na identificação de modelos e na aproximação de ganhos.
A integração de tecnologias computacionais modernas, como a programação em Python, tem aberto caminhos para a
realização desses processos, possibilitando realizar eles de forma assistida por software.
Ao empregar algoritmos e softwares especializados, é possível simplificar e agilizar as etapas tradicionais de
identificação de modelo e configuração dos controladores PID, tornando-os mais acessíveis e eficientes, especialmente
em ambientes industriais com sistemas dinâmicos.

Neste contexto, a automação e a otimização dos processos de identificação de sistemas, sintonização e modelagem de
controladores PID através de ferramentas computacionais tornam-se uma ferramenta útil, visto que podem proporcionar
uma dinâmica mais ágil a esses processos, possibilitando a análise de mais formas de identificação e parâmetros de
ganho PID, além de reduzir erros de cálculo ou aplicação de método.

\section{Justificativa}

A automatização no processo de identificação e sintonização de controladores PID oferecida por uma biblioteca em Python
pode representar um avanço interessante na eficiência e precisão desses processos.
A complexidade e o tempo necessários para a sintonização eficaz são podem ser reduzidos, e erros
associados às abordagens manuais ou menos automatizadas também podem ser minimizados.

Outra vantagem significativa é a construção da biblioteca em Python utilizando a biblioteca de
controle para Python já existente, a \textit{Python Control Systems Library} em vez de ferramentas como MATLAB.
Python, sendo uma linguagem de programação gratuita e de código aberto, oferece várias vantagens sobre o MATLAB, que é
proprietário e licenciado individualmente \cite{introart3}.
Além de ser uma opção mais econômica, o Python permite uma maior transparência e colaboração dentro da comunidade de
desenvolvedores, graças à sua natureza de código aberto.

A relevância deste projeto está em fornecer uma ferramenta de software de código aberto que possa ser utilizada na área
de controle.
A biblioteca desenvolvida pode ser uma contribuição relevante para a comunidade profissional, possivelmente promovendo
o avanço da área de controle, até mesmo de sistemas mais complexos, e facilitando a aplicação prática desses
conhecimentos em diversos setores da indústria.

Por fim, existe também o potencial significativo para uso didático com a documentação detalhada dos métodos
implementados e o acesso fácil proporcionado pela linguagem Python.
Isso pode tornar a biblioteca uma ferramenta valiosa, tanto para educadores quanto para estudantes no campo da
engenharia de controle, facilitando o aprendizado e a experimentação com conceitos de controle PID em um ambiente mais
acessível e flexível.

\section{Objetivo Geral}

O objetivo central deste trabalho é desenvolver uma biblioteca em Python para análise
e controle de sistemas de primeira ordem, visando agilizar e facilitar o processo de
análise e desenvolvimento de sistemas de controle.

\section{Objetivos Específicos}

Para atingir o objetivo principal, os seguintes itens se fazem importantes e necessários:
\begin{alineas}
    \item Criar uma biblioteca em python disponível para instalação em ambientes Windows e Linux;
    \item Utilizar uma arquitetura de código compreensível e escalável, a fim de facilitar implementações futuras
    que aumentem as funcionalidades da biblioteca;
    \item Usar a biblioteca de controle para Python já existente nas implementações, possibilitando o uso dos modelos
    criados para fins não previstos pela biblioteca;
    \item Projetar a biblioteca com facilidade de uso em mente, para que mesmo estudantes ou pessoas pouco instruídas
    em ferramentas da linguagem Python possam fazer uso da biblioteca;
    \item Desenvolver métodos de identificação de modelo da planta, com o mínimo de ações e informações necessárias por
    parte do usuário;
    \item Implementar funções para obtenção de ganhos de controle PID aproximados, de forma automática ou simplificada,
    com base no modelo da planta e em metas de desempenho do sistema;
    \item Fornecer ferramentas para visualização dos dados, desempenho e análise gráfica dos resultados;
    \item Documentar o uso e funcionamento da biblioteca de forma compreensível e acessível.
\end{alineas}
